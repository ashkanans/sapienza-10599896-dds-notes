%%%%%%%%%%%%%%%%%%%%%%%%%%%%%%%%%%%%%%%%%%%%%%%%%%%%%%%%%%%%%%
%                                                            %
%                    Read constant commands                  %
%                                                            %
%%%%%%%%%%%%%%%%%%%%%%%%%%%%%%%%%%%%%%%%%%%%%%%%%%%%%%%%%%%%%%

% Logo 

\newcommand{\ddsFiveTwelve}{resources/images/logo/distributed}

% Chapter 1

\newcommand{\timingSummary}{resources/Section 1/Summary on Timing Assumptions}
\newcommand{\linkAbstraction}{resources/Section 1/The link abstraction and different instances}
\newcommand{\interfaceFLL}{resources/Section 1/Interface of fair-loss point-to-point links}
\newcommand{\interfaceSL}{resources/Section 1/Interface of stubborn point-to-point links}
\newcommand{\interfacePL}{resources/Section 1/Interface of perfect point-to-point links}
\newcommand{\interfaceLPL}{resources/Section 1/Interface of logged perfect point-to-point links}
\newcommand{\interfaceAPL}{resources/Section 1/Interface of authenticated perfect point-to-point links}
\newcommand{\algorithmRetransmitForever}{resources/Section 1/Algorithm 2.1 - Retransmit Forever}
\newcommand{\algorithEliminateDuplicatesm}{resources/Section 1/Algorithm 2.2 - Eliminate Duplicates}
\newcommand{\algorithmLogDelivered}{resources/Section 1/Algorithm 2.3 - Log Delivered}

\input{resources/constants/images path/IMAGE_NAMES.tex}
\input{resources/constants/images path/IMAGE_REFS.tex}
% User manual configuration

\newcommand{\userManualSimpleTextSize}{13pt}
\newcommand{\userManualSimpleTextStyle}{Times New Roman}

% This command defines custom font sizes for section, subsection, and subsubsection headings in LaTeX, and sets them accordingly using the titlesec package.
% Usage: Simply call the \myheadingstyles command in the preamble of your LaTeX document to define and set the custom font sizes.
% Arguments: None.

\newcommand{\myheadingstyles}{
	\newcommand{\sectionfont}{\fontsize{20}{24}\selectfont}
	\newcommand{\subsectionfont}{\fontsize{18}{22}\selectfont}
	\newcommand{\subsubsectionfont}{\fontsize{16}{20}\selectfont}
	
	\titleformat{\section}
	{\sectionfont\bfseries}{\thesection}{1em}{}
	\titleformat{\subsection}
	{\subsectionfont\bfseries}{\thesubsection}{1em}{}
	\titleformat{\subsubsection}
	{\subsubsectionfont\bfseries}{\thesubsubsection}{1em}{}
}

\newcommand{\myTOC}{
	% Set font size and style for the table of contents
	\renewcommand{\contentsname}{\fontsize{14}{16}\selectfont\textbf{\fontfamily{ptm}\selectfont Table of Contents}}
	\Large
	% Print the table of contents
	\tableofcontents
	
	\newpage
	
	% Set font size and style for the list of figures
	\renewcommand{\listfigurename}{\fontsize{14}{16}\selectfont\textbf{\fontfamily{ptm}\selectfont List of Figures}}
	\Large
	\listoffigures
	
	\newpage
	
	% Set font size and style for the list of tables
	\renewcommand{\listtablename}{\fontsize{14}{16}\selectfont\textbf{\fontfamily{ptm}\selectfont List of Tables}}
	\Large
	\listoftables
	
	\newpage
}

%%%%%%%%%%%%%%%%%%%%%%%%%%%%%%%%%%%%%%%%%%%%%%%%%%%%%%%%%%%%%%
%                                                            %
%                       Custom Commands                      %
%                                                            %
%%%%%%%%%%%%%%%%%%%%%%%%%%%%%%%%%%%%%%%%%%%%%%%%%%%%%%%%%%%%%%

% Usage: \imageCaptionTable{image path}{image caption}{image label}{text}
% - The first argument should be the path to the image file in any format.
% - The second argument should be the caption of the image to be displayed.
% - The third argument should be the label of the image.
% - The fourth argument should be the text to be displayed on the right side of the image.
% - The resulting table will have the image displayed on the left (with the caption and label) and the text on the right.

\newcommand{\imageCaptionTable}[4]{%
	\begin{tabular}{l l}
		\includegraphics[width=0.5\textwidth, caption={#2}, label={#3}]{#1} & \centering
		\begin{varwidth}[t]{0.8\textwidth}
			\configuratedText{#4}
		\end{varwidth}
	\end{tabular}
}

% Inserts a small image in the middle of a text
% Usage: \smallimage[optional scale]{image file name}
% - The optional scale argument (default 0.5) can be used to adjust the size of the image.
% - The image file should be in the same directory as the LaTeX document.

\newcommand{\smallimage}[2][0.03]{
	\includegraphics[width=#1\linewidth]{#2}
}


% Creates a new figure with an image centered on the page, and adds a caption and label for reference.
% Usage: \sectionCenteredfigure[optional scale]{image file name}{caption text}{label}
% - The optional scale argument (default 0.9) can be used to adjust the width of the image.
% - The image file should be in the same directory as the LaTeX document.
% - The caption text should describe the contents of the image.
% - The label should be a unique identifier for the figure, used for referencing it later in the text.

\newcommand{\sectionCenteredfigure}[4][0.9]{
	\begin{figure}[H]
		\centering
		\fbox{\includegraphics[width=#1\linewidth]{#2}}
		\caption{#3}
		\label{fig:#4}
	\end{figure}
}


% Creates a new block of justified text with a specified font and font size.
% Usage: \generalText{font family}{font size}{text}
% - The font family argument specifies the font to be used (e.g., Times New Roman, Arial, etc.).
% - The font size argument specifies the size of the font (e.g., 12pt, 14pt, etc.).
% - The text argument should contain the text to be justified.
% - The resulting block of text will be fully justified (i.e., aligned with both the left and right margins).

\newcommand{\customText}[3]{%
	\par\begingroup
	\setlength{\parindent}{0pt}%
	\linespread{1.3}%
	\fontsize{#2}{#2}%
	\fontfamily{#1}\selectfont #3%
	\par\endgroup%
}

% Creates a new block of justified text with the font size and font style of configuration file.
% Usage: \generalText{text}
% - The font family argument is specfied by \userManualSimpleTextStyle 
% - The font size argument is specified by \userManualSimpleTextSize
% - The text argument should contain the text to be justified.
% - The resulting block of text will be fully justified (i.e., aligned with both the left and right margins).

\newcommand{\configuratedText}[1]{%
	\par\begingroup
	\setlength{\parindent}{0pt}%
	\linespread{1.3}%
	\selectfont #1%
	\par\endgroup%
}

% Defines a new command for referencing figures.
% Usage: \figref{label}
% - The argument should be the label of the figure to be referenced.
% - The resulting output will be in the format "(Figure <number>)", where <number> is the number of the referenced figure.
% - The label should be defined using \label{fig:<label>} command in the figure environment.
% - The \hyperref command creates a hyperlink to the referenced figure.
% - The \ref* command produces only the number of the referenced figure, without the preceding "Figure" text.

\newcommand{\figref}[1]{(\hyperref[fig:#1]{Figure \ref*{fig:#1}})}

% Creates a new table with an icon and its name.
% Usage: \customTable{icon path}{icon name}
% - The first argument should be the path to the icon file in PNG format.
% - The second argument should be the name of the icon to be displayed.
% - The resulting table will have the icon displayed on the left and its name on the right.

\newcommand{\iconNameTable}[2]{%
	\begin{tabular}{l l}
		\includegraphics[width=0.03\textwidth]{#1} & \centering 
		\begin{varwidth}[t]{0.8\textwidth}
			\configuratedText{#2}
		\end{varwidth}
	\end{tabular}
}

% Creates a new table with an icon and its name.
% Usage: \customTable{icon path}{icon name}
% - The first argument should be a text.
% - The second argument should be a text.
% - The resulting table will have the text displayed on the left and a text on the right.

\newcommand{\textTextTable}[3][2cm]{%
	\begin{tabular}{p{#1} p{\dimexpr0.90\textwidth-#1}}
		\configuratedText{#2}
		&
		\begin{varwidth}[t]{\linewidth}
			\configuratedText{#3}
		\end{varwidth}
	\end{tabular}
}

% Creates a new table with an icon, its name, and its description.
% Usage: \iconNameDescriptTable{icon path}{icon name}{icon description}
% - The first argument should be the path to the icon file in PNG format.
% - The second argument should be the name of the icon to be displayed.
% - The third argument should be a description of the icon.
% - The resulting table will have the icon displayed on the left, its name in the middle, and its description on the right.
% - The table has three columns with widths of 0.1, 0.3, and 0.5 times the text width, respectively.
% - The second and third columns are aligned to the left.

\newcommand{\iconNameDescriptTable}[3]{%
	\begin{tabular}{p{0.05\textwidth} p{0.2\textwidth} m{0.6\textwidth}}
		\includegraphics[width=0.03\textwidth]{#1} & \raggedright \configuratedText{#2} & \justify \configuratedText{#3} \
	\end{tabular}
}

% Creates a new table with an text, its name, and its description.
% Usage: \textDescriptTable{text}{text name}{text description}
% - The first argument should be the text to be displayed on the left.
% - The second argument should be the name of the text.
% - The third argument should be a description of the text.
% - The resulting table will have the text displayed on the left, its name in the middle, and its description on the right.
% - The table has three columns with widths of 0.1, 0.3, and 0.5 times the text width, respectively.
% - The second and third columns are aligned to the left.

\newcommand{\textDescriptTable}[3]{%
	\begin{tabular}{m{0.1\textwidth} m{0.1\textwidth} m{0.5\textwidth}}
		\raggedright \configuratedText{#1} & \raggedright \configuratedText{#2} & \raggedright \configuratedText{#3} \
	\end{tabular}
}

% Creates a new block of two columns with an image on the left and justified text on the right.
% Usage: \twocolumns{optional scale}{image file name}{caption text}{font family}{font size}{text}

% - 1: The optional scale argument (default 0.9) can be used to adjust the width of the image.
% - 2: The image file should be in the same directory as the LaTeX document.
% - 3: The caption text should describe the contents of the image.
% - 4: The font family argument specifies the font to be used (e.g., Times New Roman, Arial, etc.).
% - 5: The font size argument specifies the size of the font (e.g., 12pt, 14pt, etc.).
% - 6: The text argument should contain the text to be justified.

\newcommand{\twoColumns}[6]{
	\begin{minipage}[t]{0.50\textwidth}
		\customText{#4}{#5}{#6}
	\end{minipage}\hfill
	\begin{minipage}[r]{0.45\textwidth}

	\end{minipage}
}

\documentclass[\userManualSimpleTextSize]{article}
\usepackage[utf8]{inputenc}
\usepackage{graphicx}
\usepackage{geometry}
\usepackage{tocloft}
\usepackage{booktabs}
\usepackage{fancyhdr}
\usepackage{hyperref}
\usepackage{xcolor}
\usepackage{times}
\usepackage{wrapfig}
\usepackage{array}
\usepackage{varwidth}
\usepackage{float}
\usepackage{titlesec}
\usepackage{ragged2e}

% Use the new command to set the font sizes and titleformat settings
\myheadingstyles

% Customize hyperlinks in the document
\hypersetup{
	colorlinks=true, % enable colored hyperlinks
	linkcolor=black, % set the color of internal links to black
	filecolor=magenta, % set the color of links to local files to magenta
	urlcolor=blue, % set the color of links to URLs to blue
	bookmarks=true, % enable the creation of bookmarks in the PDF file
}

\geometry{
	a4paper, % set paper size to A4
	left=2cm, % set left margin to 2cm
	right=2cm, % set right margin to 2cm
	top=2.5cm, % set top margin to 2.5cm
	bottom=2.5cm % set bottom margin to 2.5cm
}


%%%%%%%%%%%%%%%%%%%%%%%%%%%%%%%%%%%%%%%%%%%%%%%%%%%%%%%%%%%%%%
%                                                            %
%                    Document Starts Here                    %
%                                                            %
%%%%%%%%%%%%%%%%%%%%%%%%%%%%%%%%%%%%%%%%%%%%%%%%%%%%%%%%%%%%%%

\begin{document}
	
	\begin{center}
		\begin{figure}[h]
			\centering
			\includegraphics[width=0.4\linewidth]{\ddsFiveTwelve}
		\end{figure}
		\vspace{1cm} 
		{\fontsize{28}{34}\selectfont \textbf{Dependable Distributed Systems}}
	\end{center}

	\vspace{1cm} 
	
	\begin{center}
	{\fontsize{22}{28}\selectfont \textbf{Professor: S. Bonomi}}
	\end{center}

	\vspace{1cm} 

	\textcolor{blue!60!black}{\rule{\linewidth}{2pt}}
	
	\vspace{10cm} 
	
	\begin{center}
	\textbf{A.Y. 2022/23} 
	\end{center}

	\thispagestyle{empty}
	
	\newpage
	
	\myTOC
		
	\newpage

	\section{Modeling Distributed Systems}
	
	\subsection{Modeling Processes and their interactions}
	
	\subsection{Specification in terms of Safety and Liveness Property}
	
	\subsection{Modeling Failures}
	
	\sectionCenteredfigure{\timingSummary}{Summary on Timing Assumptions}{summary-on-timing-assumptions}
	
	\subsection{Abstracting Communication}
	\configuratedText{The abstraction of a \textit{link} is used to represent the network components of the distributed system.\\
	Every pair of processes is connected by a bidirectional link, a topology that provides full connectivity among the processes.\\
	Concrete examples of such architectures are illustrated in \figref{the-link-abstraction-and-different-instances} include the use of (a) a fully connected mesh, (b) a broadcast medium, (c) a ring, (d) a mesh of links interconnected with bridges}
	
	\sectionCenteredfigure{\linkAbstraction}{The link abstraction and different instances}{the-link-abstraction-and-different-instances}
	
	\subsubsection{Abstracting Link Failures}
	\configuratedText{Here we will introduce the following link abstractions considering processes faults:
	\begin{itemize}
	\item 	Fair-Loss Links
	\item 	Stubborn Links
	\item 	Perfect Links
	\item 	Logged Perfect Links
	\item 	Authenticated Perfect Links
	\end{itemize}
	}
	
	\subsubsection{Fair-Loss Links}
	
	\configuratedText{The \textbf{weakest} variant of the link abstraction.}
	
	\sectionCenteredfigure{\interfaceFLL}{Interface of fair-loss point-to-point links}{interface-of-fair-loss-point-to-point-links}
	
	\subsubsection{Stubborn Links}
	
	\configuratedText{The stubborn delivery property causes every message sent over the link to be delivered at the receiver an unbounded number of times.}
	
	\sectionCenteredfigure{\interfaceSL}{Interface of stubborn point-to-point links}{interface-of-stubborn-point-to-point-links}
	
	\configuratedText{\textbf{Example: Algorithm 2.1 (Retransmit Forever)}}
	
	\sectionCenteredfigure{\algorithmRetransmitForever}{Algorithm 2.1 - Retransmit Forever}{algorithm-2-1}
	
	\configuratedText{The algorithm keeps retransmitting all messages. Here we will consider \textit{Correctness} and \textit{Perfromance}.\\ 
	\textit{\textbf{Correctness}}: for an algorithm implementing \textit{StubbornPointToPointLinks} and using \textit{FairLossPointToPointLinks} upon sl-sending a message by a correct process, the message is \textit{fll}-delivered infinitely often by the target process. Because the algorithm keeps \textit{fll}-sending those messages infinitely.\\
	The \textit{no-creation} property is also preserved by the underlying links.\\
	\textbf{\textit{Performance}}: The algorithm is not efficient. There are two ways to make it better:
	\begin{itemize}
		\item 	When the algorithm has ended they use of the stubborn links, there is no need to keep on sending the messages.
		\item 	Add an acknowledgment mechanism which notifies the sender process to stop sending the messages as they are delivered to the target process
	\end{itemize}

	}
	
	\subsubsection{Perfect Links}
	\configuratedText{With the stubborn links abstraction, it is up to the target process to check whether a given message has already been delivered or not. Adding mechanisms for detecting and suppressing message duplicates, in addition to mechanisms for message retransmission, allows us to build an even higher-level primitive: the \textbf{perfect links} abstraction, sometimes also called the \textit{reliable links} abstraction.}

	\sectionCenteredfigure{\interfacePL}{Interface of perfect point-to-point links}{interface-of-perfect-point-to-point-links}

	\configuratedText{\textbf{Example: Algorithm 2.2 (Eliminate Duplicates)}}
	
	\sectionCenteredfigure{\algorithEliminateDuplicatesm}{Algorithm 2.2 - Eliminate Duplicate}{algorithm-2-2}
	
	\configuratedText{\textit{\textbf{Correctness}}: according to the algorithm, a process p that pl-sends a message m to target process q. Assuming both these processes are correct and based on the \textit{sl} property, q eventually sl-delivers m, at least once, hence pl-delivers m. So, the reliable delivery is guaranteed. The \textit{no-duplication} property comes from the fact that the messages are being checked if they are delivered before or not and then they are pl-delivered. \textit{No creation} property follows the \textit{no creation} property of the underlying stubborn layer.\\
	\textbf{\textit{Performance}}: Maintaining physical memory for the ever growing set of messages delivered at every process. Notice that having an periodically acknowledgment and promising not to send those messages anymore is not sufficient as a message can be in transit and uppon delivery, it violates the \textit{no creation} property.
	}

	\subsubsection{Logged Perfect Links}
	
	\configuratedText{Suitable for crash-recovery process abstraction.\\ We log the output in a stable storage, which is accessed through \textit{store} and \textit{retrieve} operations.\\ Notice that perfect point-to-point links abstraction uses crash-stop processes, where a process may crash only once and a correct process never crashes. But the crash-recovery process abstraction, as used by logged perfect links, may crash a finite number of times and still called \textit{correct} only if it recovers from a crash.}
	\sectionCenteredfigure{\interfaceLPL}{Interface of logged perfect point-to-point links}{interface-of-logged-perfect-point-to-point-links}
	
	\configuratedText{\textbf{Example: Algorithm 2.3 (Log Delivered)}\\ 
	It is direct adaptation of \textit{Eliminate Duplicates} that implements perfect point-to-point links from stubborn links. It keeps a record of all the messages that have been log-delivered in the past. It stores this record in stable storage and exposes it also to the upper layer.}
	
	\sectionCenteredfigure{\algorithmLogDelivered}{Algorithm 2.3 - Log Delivered}{algorithm-2}
	
	\configuratedText{\textit{\textbf{Correctness}}: here is just like \textit{Eliminate Duplicates} algorithm. except that delivering here means logging the messages into a stable storage.\\
	\textit{\textbf{Performance}}: In terms of messages it is like \textit{Eliminate Duplicates} algorithm. However, here we require one log operation every time a new message is received.}
	
	\subsubsection{Authenticated Perfect Links}	
	
	\configuratedText{In both fair-loss and stubborn point-to-point links abstractions \textit{no creation} property can not be guaranteed in presence of Byzantine Processes.\\ Authenticated Perfect Link uses the same interface as the other point-to-point links abstractions. \textit{Authenticity} property is stronger that \textit{no creation} property. }
	\sectionCenteredfigure{\interfaceAPL}{Interface of Authenticated perfect point-to-point links}{interface-of-authenticated-perfect-point-to-point-links}

	\configuratedText{\textbf{Example: Algorithm 2.4 (Authenticate and Filter)}\\ 
	Uses a MAC to implement authenticated perfect point-to-point link over a stubborn links abstraction. }
	
	\sectionCenteredfigure{\algorithmAuthenticateFilter}{Algorithm 2.4 - Authenticate and Filter}{algorithm-2-4}
	
	\configuratedText{\textit{\textbf{Correctness}}: The \textit{reliable delivery} and \textit{no duplication} follows the same argument as for algorithm 2.2. For \textit{authenticity} property\\
	\textit{\textbf{Performance}}: The same issue arises for the set \textit{delivered} that grows without bound. Cryptographic authentication adds a modest computational overhead.}

	\newpage
	\section{Time in Distributed Systems}
	
	\configuratedText{An important aspect of a distributed system characterization is the behavior of its processes and links with respect to time. Here, we first introduce time-related models and then consider the failure-detector abstraction as a useful way to abstract timing assumptions.}
	\newpage
	\section{Logical Clock}
	
	\newpage
	\section{Distributed Mutual Exclusion}
	
	\newpage
	\section{Broadcast Communications}
	
	\newpage
	\section{Consensus}
	
	\newpage
	\section{Ordered Communications}
	
	\newpage
	\section{Registers}
	
	\newpage
	\section{Software Replication}
	
	\newpage
	\section{Overview on Capacity Planning}
	
	\newpage
	\section{Modeling the Workload of a System}
	
	\newpage
	\section{Building a Performance Model 1}
	
	\newpage
	\section{Building a Performance Model 2}
	
	\newpage
	\section{Dependability Evaluation}
	
	\newpage
	\section{Intro to Experimental Design}
	
	\newpage
	\section{CAP Theorem}
	
	\newpage
	\section{Consistency Criteria for Distributed Shared Memories}
	
	\newpage
	\section{Publish-Subscribe Communication Paradigm}
	
	\newpage
	\section{Overlay Networks}
	
	\newpage
	\section{DLT and Blockchain}
		
	\newpage
	\section{Exercises}
	\emph{Notice: The exercises are from 2022-2023 academic year.}
	
	\newpage
	\section{Exams}

	
\end{document}\dfrac{num}{den}